\section{Future works}\label{future-works}

\subsection{Hub Server}

\paragraph{Eager Push / Lazy Pull Gossip.}
The current gossip protocol relies exclusively on eager push: every message is forwarded immediately to a random subset of peers.
If a message is lost (UDP drop, temporary peer unavailability), there is no recovery mechanism.
A hybrid \textit{eager push / lazy pull} approach would improve reliability without sacrificing the AP model: hubs would periodically exchange compact state digests (e.g., per-origin nonce vectors), and request missing updates from peers that advertise newer state.
This adds convergence guarantees while keeping the protocol fully asynchronous and partition-tolerant.

\paragraph{Gossip Protocol Authentication.}
As discussed in \cref{security}, gossip messages are currently unauthenticated.
A practical improvement would be to add HMAC-based message signing using a shared secret distributed via Kubernetes Secrets.
Each gossip message would include a signature computed over its serialized payload, and receiving hubs would discard messages with invalid signatures.
This prevents gossip injection attacks without requiring a full PKI infrastructure.

\paragraph{Weighted Room Assignment.}
The current matchmaking logic returns the first joinable room found in the local state.
A smarter strategy would consider room fill level, geographic proximity (if hubs are distributed across regions), and hub load.
For example, preferring rooms that are closer to being full (to consolidate players and free up resources) or rooms owned by the local hub (to minimize cross-hub coordination).

\paragraph{Graceful Hub Scaling.}
Currently, scaling the cluster requires updating the \texttt{EXPECTED\_HUB\_COUNT} in the ConfigMap and restarting the StatefulSet.
An improvement would be to make the expected hub count dynamic: new hubs joining the cluster would announce themselves via \texttt{PEER\_JOIN}, and existing hubs would adjust their discovery behavior automatically.
This would enable Kubernetes Horizontal Pod Autoscaler integration, scaling the hub cluster based on matchmaking request rate.

\paragraph{Gossip Protocol Metrics.}
Adding observability to the gossip layer would aid debugging and performance tuning in production.
Useful metrics include: messages sent/received per second, duplicate message rate, average propagation latency (time from origin to last hub), peer churn rate, and fanout utilization.
These could be exposed via a Prometheus-compatible \texttt{/metrics} endpoint and visualized in Grafana.

\subsection{Room Server}

\paragraph{Spectator Mode.}
Currently, only active players can connect to a room.
Adding a spectator mode would allow additional clients to observe the game in real time without participating.
Spectators would receive the same state broadcasts as players but would be excluded from the game engine's input processing.

\paragraph{Persistent Game History.}
The \texttt{GameStatePersistence} module currently handles in-memory state snapshots.
Extending it with a lightweight persistent backend (e.g., SQLite per room) would enable game replay, player statistics, and leaderboards across sessions.

\paragraph{Dynamic Map Generation.}
The current game map is fixed.
A procedural map generator that varies wall placement, map size, and power-up distribution based on player count would improve replayability.

\subsection{General}

\paragraph{End-to-End Encryption.}
Currently, all communication exception for client-hub is unencrypted within the cluster.
We could provide mutual TLS between all components.

\paragraph{Client Reconnection.}
If a client disconnects during a game (network failure, crash), there is currently no way to rejoin the same room.
Implementing a session token mechanism, where the hub assigns a token at matchmaking time and the room accepts reconnections bearing that token within a grace period, would improve the player experience on unstable networks.

\paragraph{Cross-Region Deployment.}
The current deployment assumes a single Kubernetes cluster.
Extending the gossip protocol to span multiple clusters (e.g., one per geographic region) would enable global matchmaking with region-aware room assignment.
This would require gossip messages to traverse inter-cluster links (e.g., via a dedicated gateway hub per region) and a latency-aware room selection strategy.