\begin{abstract}
    Up to $\sim$2000 characters briefly describing the project.
\end{abstract}

\section*{Disclaimer (if needed)}

During the preparation of this work, the author(s) used [NAME TOOL / SERVICE] to [REASON].
After using this tool/service, 
the author(s) reviewed and edited the content as needed 
and take(s) full responsibility for the content of the final report/artifact.

\section*{Quick \LaTeX{} suggestions}

If you need to cite a reference, you can use the \texttt{cite} command, 
like providing the BibTex key of some entry in the \texttt{references.bib} file, 
e.g.: \cite{adams1995hitchhiker}.
You can find pre-coocked BibTex entries for most Computer Science papers on \href{https://dblp.uni-trier.de/}{DBLP}.
If you need to include an image,
let \LaTeX{} decide where to put it by using the \texttt{figure} environment,
with a \texttt{includegraphics} command inside it.
%
If your need to reference a figure,
use the \texttt{label} command to assign a label to the figure,
and then use the \texttt{cref} command to reference it.
%
Put figures in the \texttt{figures} folder.
%
A complete example is shown in \cref{fig:universe}.
\begin{figure}
    \centering
    \includegraphics[width=0.5\textwidth]{figures/universe.jpg}
    \caption{This is an example image}
    \label{fig:universe}
\end{figure}

Do \textbf{not} put any placement constraint on figures,
such as \texttt{[h]} or \texttt{[h!]}.
%
Same considerations apply for other floating elements 
(e.g. tables, algorithms, listings, etc.).
Do \textbf{not} use the \texttt{\textbackslash\textbackslash} or \texttt{\textbackslash newline} commands to break lines.
Just leave an empty line between two paragraphs to start a new one.
%
The new line will be automatically indented.
%
This is intended: it is the \LaTeX{} way to separate capoverses.