\section{User Guide}\label{user-guide}

\subsection{Starting the Client}

Ensure the Python environment is activated, then launch the client from the project's root directory using the following command:

\begin{verbatim}
python3 Client.py
\end{verbatim}

The client automatically contacts the Hub Server via the \texttt{/matchmaking} endpoint to obtain a room assignment.
This process is transparent to the user: the client resolves the room address and establishes a TCP connection without manual configuration.

\subsection{Joining a Game}

Upon connection, the client prompts for a player identifier, as shown in \cref{fig:client-player-id}.

\begin{figure}[H]
    \centering
    \includegraphics[width=0.5\textwidth]{figures/client-player-id.png}
    \caption{The client prompts the user to enter a Player ID.}
    \label{fig:client-player-id}
\end{figure}

After entering a Player ID, the client connects to the assigned room.
The game map is displayed in an ASCII grid, and the interface shows the current game state along with the number of players needed to start (\cref{fig:client-waiting}).

\begin{figure}[H]
    \centering
    \includegraphics[width=0.5\textwidth]{figures/client-waiting.png}
    \caption{The player has joined the room and is waiting for other players. The map is visible, with the player's character placed on the grid.}
    \label{fig:client-waiting}
\end{figure}

\subsection{Game Start}

When a second player joins the room, a countdown timer begins, as shown in \cref{fig:client-countdown}.
The countdown gives additional players time to join before the game starts.

\begin{figure}[H]
    \centering
    \includegraphics[width=0.5\textwidth]{figures/client-countdown.png}
    \caption{A second player (F) has joined. The countdown to game start is displayed.}
    \label{fig:client-countdown}
\end{figure}

The countdown duration decreases as more players join.
Once four players are connected, the countdown is shortened and the game starts quickly.

\subsection{Controls}

During gameplay, the player controls their character using the keyboard:

\begin{center}
    \begin{tabular}{c l}
        \textbf{Key} & \textbf{Action} \\
        \hline
        \texttt{W} & Move up \\
        \texttt{A} & Move left \\
        \texttt{S} & Move down \\
        \texttt{D} & Move right \\
        \texttt{E} & Place bomb \\
        \texttt{Q} & Quit the game \\
    \end{tabular}
\end{center}

The objective is to eliminate all other players by placing bombs.
Bombs explode after a short delay, destroying breakable walls (represented by \texttt{+}) and eliminating players caught in the blast radius (represented by \texttt{*}).
The \texttt{\#} symbols represent indestructible walls.

\subsection{Game Over}

When only one player remains, the game ends and the winner is announced (\cref{fig:client-gameover}).

\begin{figure}[H]
    \centering
    \includegraphics[width=0.7\textwidth]{figures/client-gameover.png}
    \caption{The game has ended. The winner is displayed, and the server announces a reset.}
    \label{fig:client-gameover}
\end{figure}

After the game ends, the server resets and the client disconnects.
The terminal returns to the working directory.
To play again, the user must restart the client with \texttt{python3 Client.py}.